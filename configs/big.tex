
\documentclass[oneside,final,8pt,]{extreport}

\usepackage[T2A]{fontenc}
\usepackage[utf8]{inputenc}
\usepackage[russian]{babel}
\usepackage{csquotes}
\usepackage{vmargin}
\usepackage{amsmath}
\usepackage{amsfonts}
\usepackage{amssymb}
\usepackage{listings}
\usepackage{xurl}
\usepackage{enumitem}
\usepackage{titlesec}
\usepackage{needspace}
\usepackage{algorithm}
\usepackage{algorithmic}
\raggedbottom

\titleformat{\section}{\normalfont\Large\bfseries}{\thesection}{1em}{}
\usepackage{makecell}  % для переноса строк в таблицах
\usepackage[backend=biber, style=authortitle]{biblatex}  % для ссылок на литературу
\addbibresource{references.bib}

\usepackage{hyperref}  % для ссылок и цвета цитирований
\hypersetup{
    colorlinks=true,
    linkcolor=blue,
    citecolor=magenta,
    filecolor=magenta,      
    urlcolor=cyan,
    pdftitle={Overleaf Example},
    pdfpagemode=FullScreen,
}

\usepackage{graphicx}
\usepackage{wrapfig}

\setpapersize{A5}  % for big versiom -- A4, for small -- A2
\setmarginsrb{1cm}{1.5cm}{1cm}{1.5cm}{0pt}{0mm}{0pt}{13mm}
\usepackage{indentfirst}
\sloppy
\usepackage{multicol}
\usepackage{tcolorbox}
\setlength{\columnseprule}{1pt}

\DeclareGraphicsExtensions{.pdf,.png,.jpg}

% \newcommand{\todo}[1]{\textcolor{red}{#1}}
\newcommand{\todo}[1]{\colorbox{red}{#1}}

\newenvironment{proof}  % док-во теорем
    {$\blacktriangle$}  % without '$' does not work
    {$\blacksquare$}
    
\newcommand{\mathLet}{\scalebox{1.}[1.5]{$\sqsupset$}}  % math symbol 'let'

\usepackage{fontawesome}  % to import '\faEye' -- math symbol 'eye'
% \newcommand{\maeye}{\scalebox{2.}{\faEye}}  % math symbol 'eye', very big

%\usepackage{pst-optic}

\newcommand*\latseye{%
       \scalebox{0.25}{\begin{pspicture}(-1,-1)(1,1)
\rput(1,-.5){\eye}
\end{pspicture}\kern1em}}

\newcommand*\latdeye{%
       \reflectbox{\scalebox{0.25}{\begin{pspicture}(-1,-1)(1,1)
\rput(1,-.5){\eye}
\end{pspicture}\kern.2em}}}

\everymath{\displaystyle}
\setlength\parindent{0pt}

% \setlength{\leftmargini}{10pt}  % чтоб отступы в итемайзах были поменьше 

\newlist{myitems}{enumerate}{3}

\setlist[myitems,1]{label = $\bullet$, leftmargin=0em, itemindent = 1.0em}
\setlist[myitems,2]{label = $-$, leftmargin=1em, itemindent = 1.0em}
\setlist[myitems,3]{label = $--$, leftmargin=2em, itemindent = 1.0em}

\renewenvironment{itemize}
{\begin{myitems}
\setlength{\itemsep}{0pt}}
{\end{myitems}}


\newlist{myitems2}{enumerate}{3}

\setlist[myitems2,1]{label*=\arabic*.,leftmargin=0.5em, itemindent = 1.0em}
\setlist[myitems2,2]{label*=\arabic*.,leftmargin=1.5em, itemindent = 1.0em}
\setlist[myitems2,3]{label*=\arabic*.,leftmargin=2.5em, itemindent = 1.0em}


\renewenvironment{enumerate}
{\begin{myitems2}
\setlength{\itemsep}{0pt}}
{\end{myitems2}}

\newcommand*{\nsection}[1]{
    \section*{#1}
    \addcontentsline{toc}{section}{#1}
}