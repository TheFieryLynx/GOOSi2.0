% БОЛЬШАЯ КНИЖНАЯ ВЕРСИЯ

\documentclass[oneside,final,8pt,]{extreport}

\usepackage[T2A]{fontenc}
\usepackage[utf8]{inputenc}
\usepackage[russian]{babel}
\usepackage{csquotes}
\usepackage{vmargin}
\usepackage{amsmath}
\usepackage{amsfonts}
\usepackage{amssymb}
\usepackage{listings}
\usepackage{enumitem}
\usepackage{titlesec}
\usepackage{needspace}
\usepackage{algorithm}
\usepackage{algorithmic}
\raggedbottom

\titleformat{\section}{\normalfont\Large\bfseries}{\thesection}{1em}{}
\usepackage{makecell}  % для переноса строк в таблицах
\usepackage[backend=biber, style=authortitle]{biblatex}  % для ссылок на литературу
\addbibresource{references.bib}

\usepackage{hyperref}  % для ссылок и цвета цитирований
\hypersetup{
    colorlinks=true,
    linkcolor=blue,
    citecolor=magenta,
    filecolor=magenta,      
    urlcolor=cyan,
    pdftitle={Overleaf Example},
    pdfpagemode=FullScreen,
}

\usepackage{graphicx}
\usepackage{wrapfig}

\setpapersize{A5}  % for big versiom -- A4, for small -- A2
\setmarginsrb{1cm}{1.5cm}{1cm}{1.5cm}{0pt}{0mm}{0pt}{13mm}
\usepackage{indentfirst}
\sloppy
\usepackage{multicol}
\usepackage{tcolorbox}
\setlength{\columnseprule}{1pt}

\DeclareGraphicsExtensions{.pdf,.png,.jpg}

% \newcommand{\todo}[1]{\textcolor{red}{#1}}
\newcommand{\todo}[1]{\colorbox{red}{#1}}

\newenvironment{proof}  % док-во теорем
    {$\blacktriangle$}  % without '$' does not work
    {$\blacksquare$}
    
\newcommand{\mathLet}{\scalebox{1.}[1.5]{$\sqsupset$}}  % math symbol 'let'

\usepackage{fontawesome}  % to import '\faEye' -- math symbol 'eye'
% \newcommand{\maeye}{\scalebox{2.}{\faEye}}  % math symbol 'eye', very big

%\usepackage{pst-optic}

\newcommand*\latseye{%
       \scalebox{0.25}{\begin{pspicture}(-1,-1)(1,1)
\rput(1,-.5){\eye}
\end{pspicture}\kern1em}}

\newcommand*\latdeye{%
       \reflectbox{\scalebox{0.25}{\begin{pspicture}(-1,-1)(1,1)
\rput(1,-.5){\eye}
\end{pspicture}\kern.2em}}}

\everymath{\displaystyle}
\setlength\parindent{0pt}

% \setlength{\leftmargini}{10pt}  % чтоб отступы в итемайзах были поменьше 

\newlist{itemize2}{enumerate}{1}
\setlist[itemize2,1]{label = $\bullet$, leftmargin=0em, itemindent = 1.2em}

\renewenvironment{itemize}
{\begin{itemize2}}
{\end{itemize2}}
\renewcommand{\thesection}{\arabic{section}}
\begin{document}
\begingroup
    % \fontsize{36pt}{50pt}\selectfont
    % \large
    \centerline{ВНИМАНИЕ!}
    \centerline{спасибо за внимание}
    \centerline{\hfill\hrulefill\hrulefill\hfill}
    \vskip1.5cm
    \centerline{GOOSi 2.0}
    \centerline{Материалы для ГОСов 6 курса. Мяу.}
    \vskip1.5cm
    
    LaTeX-исходники этого материала вы можете найти здесь: \url{https://github.com/TheFieryLynx/GOOSi2.0}

    \bigbreak
    \bigbreak
    \bigbreak
    \bigbreak
    \bigbreak
    \rightline{Мальчик, водочки нам принеси. Мы МГУ закончили.}

    \vfill
    \centerline{По разные стороны Москвы -- XXI век}
\endgroup

\tableofcontents
\newpage
% -------------------------------------------- OSN -

\nsection{OSN 1-2 Основные понятия дедуктивной верификации. Методы доказательства корректности программ.} 

Целью любой верификации программы является установление соответствия программы ее требованиям. Дедуктивная верификация устанавливает это соответствие в виде логического вывода утверждения о том, что программа соответствует требованиям. При этом доказывается соответствие программы требованиям на всех входах программы.
\paragraph{Математическая модель программы}

Переменные разделяются на три типа: входные, промежуточные и выходные.

\textbf{Входные} переменные содержат исходные входные значение и никогда не меняются во время работы программы. \textbf{Промежуточные} переменные используются для хранения промежуточных результатов в процессе вычисления. \textbf{Выходные} переменные содержат значения, вычисляемые данной программой.(входные переменные будем обозначать как x1, x2, …, промежуточные как y1, y2, …, выходные как z1, z2, …).

Каждая переменная v может принимать значения из некоторого множества Dv, которое называется \textbf{доменом} переменной.

Входной домен: Dx = Dx1  Dx2  …  Dxa; Домен программы: Dy = Dy1  Dy2  …  Dyb; Выходной домен: Dz = Dz1  Dz2  …  Dzc; 

Универсальный домен D: множество значений всех возможных переменных.

Расширенный домен Dv+: домен переменной v, дополненный специальным значением \{$\omega$\}, которое не входит в универсальный домен: $D^v_+$ = Dv $\cup$ \{$\omega$\}

\paragraph{Операторы программы}
5 видов операторов программы над множеством переменных:

Начальный оператор \textbf{START}: y $\leftarrow$  f(x). Здесь f является функцией Dx $\rightarrow$ Dy+, инициализирующей промежуточные переменные программы на основе значений ее входных переменных.

Оператор присваивания \textbf{ASSIGN}: y $\leftarrow$ g(x, y). Здесь g является функцией                 Dx X Dy $\rightarrow$ $D^y_+$, вычисляющей новые значения промежуточных переменных.

Условный оператор \textbf{TEST}: t(x, y). Здесь t является предикатом на множестве значений входных и промежуточных переменных программы.

Оператор соединения  \textbf{JOIN}.

Оператор завершения  \textbf{HALT}: z $\leftarrow$ h(x, y). Здесь h является функцией Dx X  Dy $\rightarrow$ $D^z_+$, устанавливающей значения выходных переменных программы.

\includegraphics[scale=0.2]{pics/halt.png}

\textbf{Модель программы} - блок-схема. \textbf{Блок-схема} это тройка ( V, N, E ), где \textbf{V} – конечное множество переменных программы, \textbf{N} – конечное множество операторов блок-схемы, E $\subseteq$ N x \{ T, F, $\epsilon$ \} x N – конечное множество связок блок-схемы, помеченных символами T, F или $\epsilon$.

\textbf{Корректно-определенная блок-схема}:
\begin{enumerate}
    \item Ровно один начальный оператор и не менее одного завершающего оператора.
    \item Любой оператор находится на ориентированном пути от начального оператора к некоторому завершающему оператору.
    \item Число связок, выходящих из каждого оператора, и пометки этих связок соответствуют типу оператора: 
    \begin{itemize}
        \item Из начального оператора выходит ровно 1 дуга, помеченная символом $\epsilon$
        \item Из оператора присваивания выходит ровно 1 дуга, помеченная символом $\epsilon$.
        \item Из условного оператора выходит ровно 2 дуги, причем одна из них помечена символом T, а другая – символом F.
        \item Из оператора соединения выходит ровно 1 дуга, помеченная символом $\epsilon$.
        \item Из завершающего оператора не выходит ни одной дуги.
    \end{itemize}
    \item Число связок, входящих в каждый оператор, соответствует его типу
    \begin{itemize}
        \item В начальный оператор не входит ни одна дуга
        \item В оператор присваивания, условный и завершающий оператор входит ровно одна дуга
        \item В оператор соединения входит не менее одной дуги
    \end{itemize}
\end{enumerate}

\textbf{Определение}: $\forall$ оператора n и символа s в корректно-определенной блок-схеме (V, N, E) $\exists$ не более одного оператора n', что ( n, s, n' ) $\in$ E. Такой оператор n' (если он существует) мы будем называть \textit{последователем оператора} n по пометке s и обозначать как succ(n, s)

\includegraphics[scale=0.5]{pics/block_scheme.png}

\textbf{Конфигурацией программы} P будем называть пару ( \textit{l}, $\sigma$ ), где \textit{l} $\in$ $\lambda_P$ – метка текущего оператора программы, $\sigma$ =($d_1,d_2,...,d_{a+b}$ ) $\in$ $D_x^+$ x $D_y^+$ – вектор значений входных и промежуточных переменных программы.


Конечная или бесконечная последовательность конфигураций \{ $C_i$ | i=1,..,n,...\} программы P называется \textbf{вычислением}, если

\begin{enumerate}
    \item Метка первой конфигурации программы является меткой начального оператора.
    \item Значения всех входных переменных программы являются определенными ($\neq \omega$)  неизменными во всех конфигурациях вычисления.
    \item Значения промежуточных переменных в первой конфигурации равны $\omega$ (\textit{не определены}).
    \item Если метка $\textit{l}_i$ текущего оператора конфигурации $C_i$ является меткой начального оператора START: y $\leftarrow$ f(x), то следующая конфигурация $C_{i+1}$ состоит из метки оператора succ($n_i$, $\epsilon$) и вектора значений переменных $\sigma_{i+1} = \sigma_i[y \leftarrow f(x)]$.
    \item Если метка $\textit{l}_i$ текущего оператора конфигурации $C_i$ является меткой оператора присваивания ASSIGN: y $\leftarrow$ g(x, y), то следующая конфигурация $C_{i+1}$ состоит из метки оператора succ($n_i$, $\epsilon$) и вектора значений переменных $\sigma_{i+1} = \sigma_i[y \leftarrow g(x, y)]$..
    \item Если метка $\textit{l}_i$ текущего оператора конфигурации $C_i$ является меткой условного оператора TEST: t(x, y) и предикат t(x, y) при значениях переменных $\sigma_i$ принимает значение T, то следующая конфигурация $C_{i+1}$ состоит из метки оператора succ($n_i$, T) и вектора значений переменных $\sigma_{i+1} = \sigma_i$.
    \item Если метка $\textit{l}_i$ текущего оператора конфигурации $C_i$ является меткой условного оператора TEST: t(x, y) и предикат t(x, y) при значениях переменных $\sigma_i$ принимает значение F, то следующая конфигурация $C_{i+1}$ состоит из метки оператора succ($n_i$, F) и вектора значений переменных $\sigma_{i+1} = \sigma_i$.
    \item Если метка $\textit{l}_i$ текущего оператора конфигурации $C_i$ является меткой оператора соединения JOIN, то следующая конфигурация $C_{i+1}$ состоит из метки оператора succ($n_i$, $\epsilon$) и вектора значений переменных $\sigma_{i+1} = \sigma_i$.
    \item Если метка $\textit{l}_i$ текущего оператора конфигурации $C_i$ является меткой завершающего оператора HALT: z $\leftarrow$ h(x, y), то $C_i$ является последней конфигурацией вычисления.
    \item Если в конфигурации $C_{i+1}$ значение какой-либо промежуточной переменной равно $\omega$, то это последняя конфигурация вычисления.
\end{enumerate}

\textbf{Лемма}: Для каждой блок-схемы P и вектора значений ее входных переменных x существует единственное вычисление, в первой конфигурации которого значения входных переменных равны x.


\subsection{OSN 2 Основные понятия дедуктивной верификации. Методы доказательства
завершимости программ.}

\subsection{OSN 3 Основные сведения об объектном языке ограничений (OCL): состав OCL-выражения,навигация по ассоциациям, виды коллекций, операции с коллекциями, учёт наследования в выражениях и наследование ограничений. Примеры использования OCL.}

\section*{OSN 4 Способы объектно-реляционного отображения для классов и атрибутов, бинарных и N-арных ассоциаций, классов ассоциаций, иерархий наследования. Примеры применения этих способов. Моделирование схемы реляционной базы данных с помощью диаграммы классов.}

\section*{OSN 5 Образцы (паттерны) проектирования, их классификация и способ описания. Примеры образцов: структурного, поведенческого и порождающего.}

\subsection{OSN 6 Основные понятия безопасности информации: конфиденциальность,
целостность,доступность. Виды защиты информации. Модель Белла-Лападулы.
Понятие ошибки, уязвимости в программном обеспечении, примеры.}

\nsection{OSN 7 Ошибка типа «переполнение буфера». Выполнение произвольного кода на исполнимом стеке. Противодействие выполнению кода на стеке: «канарейка»,
DEP. Выполнение произвольного кода на неисполнимом стеке. Return-to-libc, return-orientedprogramming(ROP).}

\textbf{Ошибка типа «переполнение буфера»}
Программа осуществляет запись в буфер, размещенный на стеке, по неверному индексу, превышающему наибольший допустимый. Ошибка типична для языков Си и Си++, а также для ассемблера. Возможные последствия эксплуатации уязвимости: нарушение доступности — аварийное завершение или зависание программы; нарушение конфиденциальности, целостности и доступности — перехват потока управления, внедрение произвольного кода. \textbf{CWE оценивает вероятность эксплуатации как очень высокую.}

Для \textbf{эксплуатации переполнения}, атакующий пишет в буфер короткий участок исполнимого кода, который выполнит некое выбранное атакующим действие. Адрес возврата переписывается соответствующим адресом из буфера, заставляя процессор исполнять этот код при попытке возврата из процедуры.
\textit{(Язык Си никогда не проверяет выполняемые над буферами чтения и записи, допуская переполнение.)}


\textbf{Условия реализации атаки}
\begin{itemize}
    \item \textbf{Исполнимый стек:} внедряемый код размещён на стеке. Если система не позволяет выполнять код из диапазона адресов, относящегося к стеку, атака не удастся.
    \item \textbf{Относительно корректное завершение функции:} после того, как перезаписан адрес возврата и предшествующие ему значения в стеке, функция должна доработать до команды ret, чтобы выполнился внедряемый код.
    \item \textbf{Постоянные адреса: }в рамках последовательных запусков программы адреса объектов в стеке не должны меняться, т.к. иначе перезаписанный адрес возврата перестанет быть корректным, и вместо перехвата потока управления можно будет получить лишь аварийное завершение программы.
\end{itemize}

\textbf{Противодействие выполнению кода на стеке}

\textbf{Неисполнимый стек} (DEP) — технология, позволяющая помечать сегменты или страницы памяти стека как неисполняемые. При попытке передать управление на код, размещённый в такой памяти, происходит аварийное завершение процесса. Аппаратно поддерживается на уровне страничной трансляции во всех процессорах x86\_64, а также в более новых x86 и ARM. В более старых моделях x86 возможна более медленная и сопряжённая с дополнительными ограничениями на исполнимые файлы реализация с использованием сегментной трансляции.

\textbf{(DEP)} Когда системы стараются разделить память на записываемую и исполняемую, но не одновременно ту и другую. (обычно реализовано в аппаратной поддержке виртуальной памяти). 

\textbf{Обход DEP: return-to-libc}
Вместо передачи управления на код внутри буфера можно заменить адрес возврата на адрес известной библиотечной функции, например system из стандартной библиотеки Си. Она принимает один параметр (обычно через стек), адрес строки, представляющий собой команду для исполнения. Атакующий создает нужную команду и помещает адрес на неё в переполняемый буфер. Функция system(), будучи частью системной библиотеки, исполнима. Эксплойту не надо исполнять код из стека; достаточно прочитать команду с него.

\textbf{Канарейка }
Общая идея: проверять факт перезаписи стека непосредственно перед адресом возврата перед выходом из функции. «Канарейка» — как правило случайное значение, которое размещается на стеке перед адресом возврата. Перед выходом из функции происходит сравнение значения в стеке с исходным значением. Если значения не совпадают, программа аварийно завершается.

\textbf{Обход «канарейки» на стеке:}
\begin{enumerate}
    \item «Канарейка» проверяется только перед выходом из функции, однако перехват потока управления может быть осуществлён раньше (перезаписываемый указатель на функцию).
    \item «Канарейка» может быть перезаписана, если в функции есть ошибка CWE-123 (‘Write-what-where Condition’). Также в этом случае может быть перезаписан адрес возврата, а «канарейка» останется неизменной.
    \item Если программа содержит, например, ошибку CWE-126 (‘Buffer Over-read’), или возможны множественные попытки (brute force), то значение «канарейки» может быть извлечено из стека, после чего возможна эксплуатация переполнения буфера с известным значением «канарейки».
    \item Перехват обработчика исключения на Windows (SEH-эксплоит).
\end{enumerate}

\textbf{Противодействие:}
\begin{enumerate}
    \item \textbf{Рандомизация адресного пространства (ASLR)}. Делает случайной позицию стека и расположение в памяти библиотек и исполнимого кода. Они меняются при любом запуске пр-ммы, перезагрузке или их комбинации. Можно попытаться обойти защиту, используя только постоянные адреса.
    \item  \textbf{Return-oriented programming (ROP)}. \textbf{Гаджет} —  техника построения экслоита, при которой полезная нагрузка формируется в виде цепочки гаджетов с известными постоянным адресами. использует гаджеты - послед-ти команд в, заканчивающаяся командой RET. Каждый гаджет выполняет некую операцию (запись значения в регистр, сложение регистров, и т.п.)
    Для объединения гаджетов в одно целое используется длинная послед-ть адресов возврата (а также любых полезных и необходимых данных) записанных в стек в ходе переполнения буфера. Инструкции возврата прыгают с гаджета на гаджет, в то время как ф-ции вызываются редко (или никогда).
    \textbf{ Возвратно-ориентированное программирование (ROP) }— способ построения эксплоита, при котором полезная нагрузка формируется в виде цепочки гаджетов с известными постоянными адресами. 
\end{enumerate}
\nsection{OSN 8 Статический анализ исходного кода с целью поиска ошибок. Типы обнаруживаемых ошибок. Путь распространения ошибки: source, propagation, sink. Потоковая и контекстная чувствительность. Качество результата анализа: false/truepositive/negative. Интерпретация результатов анализа.}

\textbf{Анализ программы} — выявление фактов о программе;

\textbf{Статический анализ} — анализ программы без её запуска;

\textbf{Поиск ошибок} (FindBugs, Coverity, Klocwork, Svace).

\begin{itemize}
    \item \textbf{Неформальный подход} — поиск часто встречающихся ошибок:
    \begin{itemize}
        \item перекрёстная проверка кода:
        \begin{itemize}
            \item выполняется вручную;
            \item у людей похожие «слепые пятна» при просмотре кода;
        \end{itemize}
        \item анализ на уровне синтаксиса — автоматический поиск ошибочных шаблонов в коде.
    \end{itemize}
    \item \textbf{Формальный подход} — поиск всех ошибок или доказательство их отсутствия:
    \begin{itemize}
        \item верификация — формальное доказательство соответствия программы её спецификации:
        \begin{itemize}
            \item требует построения спецификации;
        \end{itemize}
        \item проверка свойств — например, <<программа не выбрасывает NullPointerException>>.
    \end{itemize}
\end{itemize}

\textbf{Ошибки работы с ресурсами:}
\begin{itemize}
    \item утечки памяти или других ресурсов;
    \item  неверная последовательность операций (например, двойное освобождение);
    \item ошибки при работе с многопоточными примитивами.
\end{itemize}

\textbf{Ошибки ввода/вывода:}
\begin{itemize}
    \item форматная строка.
\end{itemize}

\textbf{Арифметические ошибки:}
\begin{itemize}
    \item деление на ноль.
\end{itemize}

\textbf{Использование неинициализированных значений.}

\textbf{Ошибки работы с памятью:}
\begin{itemize}
    \item разыменование нулевого указателя;
    \item выход за границы буфера.
\end{itemize}

\includegraphics[width=\linewidth]{pics/quality_check.png}

\textbf{Чувствительность к пути} — способность анализа различать разные пути в программе.

\textbf{Чувствительность к потоку} — способность анализа различать порядок следования операторов.

\textbf{Чувствительность к контексту} — способность анализа различать разные вызовы одной функции.

\begin{itemize}
    \item \textbf{Цель обнаружения ошибки} — её устранение.
    \item Ошибка обнаруживается в месте её проявления в программе (выполнение некорректного действия).
    \item Исправление может требоваться в другом месте.
    \item \textbf{Путь распространения ошибки} — поток данных в программе, приведший к ошибке, делится на 3 части:
    \begin{itemize}
        \item[*] \textbf{источник (source)} — место инициализации переменных, значения которых привели к ошибке;
        \item[*] \textbf{распространение (propagation)} — операторы, участвовавшие в обработке/передаче значений, которые привели к ошибке;
        \item[*] \textbf{сток}, место проявления ошибки (sink) — оператор, приводящий к ошибке.
    \end{itemize}
\end{itemize}
\subsection{OSN 9 Применение отладки для оценки возможности эксплуатации уязвимостей. Технологии отладки. Отладка пользовательского кода. Полносистемная отладка ввиртуальной машине. Статическое и динамическое инструментирование.
Фаззинг. Разновидности фаззинга: черный ящик, белый ящик, серый ящик.}

\nsection{OSN 10 Символьное выполнение: основные понятия. Схема работы системы символьного выполнения. Предикат пути, предикат безопасности. Проблема
экспоненциального взрыва, стратегии выбора следующего состояния.}

\textbf{Символьное выполнение}
\begin{itemize}
    \item <<Выполнение>> программы не на конкретных значениях входных данных, а на символьных значениях
    \item <<Выполнение>> множества путей программы одновременно: в точке ветвления, зависящей от символьных значений происходит разделение выполнения на две ветви с добавлением ограничений из условия ветвления
    \item Технология получила быстрое развитие в последнее время благодаря росту вычислительных возможностей и появлению удобных инструментов – решатель STP, общий формат уравнений \textit{SMT-LIB2}
\end{itemize}

\textbf{Представление программы}
\begin{itemize}
    \item Программа представляется в виде бинарного дерева потенциально бесконечной глубины (циклы) – дерево символьного выполнения
    \item Вершины дерева соответствуют выполнению условных переходов
    \item Рёбра – выполнению последовательных инструкций
    \item Каждый путь в дереве описывает эквивалентный класс входных данных
    \item Формула, описывающая путь – предикат пути
\end{itemize}

\textbf{Предикат пути} - набор логических формул описывающие прохождение по данному пути выполнения

\textbf{Предикат безопасности} – набор логических формул описывающие нарушение
безопасности кода (переполнение буфера)

\textbf{Схема работы}
\begin{itemize}
    \item Обход путей программы, генерация предикатов путей
    \item При обнаружении опасной ситуации (например, деление на 0) – добавление предиката безопасности
    \item Выбор очередной формулы и отправка её решателю
    \item Если формула содержала предикат безопасности и была решена – получение набора входных данных, активирующих ошибку
\end{itemize}

\textbf{Основные проблемы подхода}
\begin{itemize}
    \item Экспоненциальный взрыв – экспоненциальный рост количества путей
    \item Моделирование окружения – обработка системных/библиотечных вызовов
    \item Ограничения решателя – сложность решения уравнений
\end{itemize}

\textbf{Стратегии выбора пути}
\begin{itemize}
    \item На основе только структуры кода:
    \begin{itemize}
        \item Обход в глубину – может <<зациклится>> в цикле
        \item Обход в ширину – очень медленно доходит до содержательного кода при наличии большого количества ветвей
    \end{itemize}
    \item  На основе покрытия кода – выбирать пути с непосещенными инструкциями, или те которые посещались меньше. Позволяет обнаруживать ошибки на редко выполняемых путях, однако может не достигать некоторых инструкций никогда.
    \item Случайный выбор. Возможности:
    \begin{itemize}
        \item Всегда выбирать путь случайно
        \item Выбирать случайно если долгое время ничего не находится (гибрид)
        \item  Выбирать случайно в случае равного приоритета путей (гибрид)
    \end{itemize}
\end{itemize}
\nsection{OSN 11 Критерии полноты тестирования. Доменные, функциональные, структурные и проблемные критерии полноты. Использование графов, грамматик
и логических выражений для построения критериев полноты тестирования. Типовые критерии покрытия кода}

\textbf{Критерии полноты тестирования} 

Набор тестов крайне важно строить так, чтобы используемые тесты проверяли как можно больше разных аспектов функциональности системы в как можно большем разнообразии ситуаций -- для этого используют критерии полноты тестирования. Чаще всего для определения критерия полноты некоторые из возможных тестовых ситуаций рассматривают как эквивалентные и определяют количество классов неэквивалентных тестовых ситуаций, встретившихся или «покрытых» во время тестирования. При этом определяется и числовая метрика тестового покрытия — доля покрытых классов ситуаций среди всех возможных. Критерий полноты может использовать различные значения метрики, например, он может требовать, чтобы полный тестовый набор всегда покрывал 100\% выделенных классов ситуаций, или же считать достаточным покрытие 85\% классов ситуаций. Поскольку для одной метрики покрытия можно определить много критериев полноты, далее речь, чаще всего, идет о различных метриках тестового покрытия.

\underline{\textit{1. Структурные критерии}} -- основанные на структуре тестируемой системы. Можно выделить три уровня структурных метрик — уровень отдельной функции или отдельного метода класса, уровень компонента или класса, включающего несколько операций, и уровень подсистемы или системы в целом, в составе которых может быть много компонентов. Для одной функции: \underline{\textit{метрика покрытия инструкций}}, равная доле выполненных во время тестирования инструкций кода функции по отношению ко всем ее достижимым инструкциям; \underline{\textit{метрика покрытия ветвей}} -- это доля покрытых в ходе теста ветвей по отношению к общему количеству достижимых ветвей (получаемые из ГрафаПУ); \underline{\textit{метрика покрытия условий}} -- оценивает выполнение (истинность и ложность) всех отдельных условий в логических выражениях, каждое условие в выражении должно быть проверено на оба возможных значения (true и false) (метрика помогает обнаруживать ошибки в условиях, которые могут быть не затронуты тестами, ориентированными только на ветвление). Для полноценного тестирования ПО рекомендуется комбинировать оба подхода, чтобы обеспечивать как выполнение всех логических ветвей программы, так и проверку всех возможных значений условий.

На уровне компонентов, содержащих несколько методов, метрики покрытия могут определяться с использованием метрик покрытия кода отдельных методов. Но кроме этого есть более высокоуровневая информация — граф вызовов методов и функций компонента друг из друга. \underline{\textit{Метрика покрытия вызовов}} вычисляется как отношение выполненных различных инструкций вызова к общему количеству достижимых таких инструкций. Метрики на основе потоков данных на уровне компонента строятся на базе изменения и использования глобальных переменных или полей класса различными методами.

На уровне подсистемы графы потоков данных и потока управления во многом сливаются, превращаясь в схемы передачи управления или сообщений между компонентами. В больших системах могут использоваться метрики, основанные просто на доле затронутых тестами функций, компонентов, форм или окон, таблиц данных или других элементов данных по отношению к общему числу соответствующих элементов (просто измеряются и не требуют для понимания определения каких-то дополнительных сущностей).

\underline{\textit{2. Доменные критерии}} -- основываются на структуре входных и выходных данных. Чтобы определить метрику покрытия входных данных, нужно разбить их на подмножества эквивалентных с какой-то точки зрения данных. Проводить разделение данных можно двумя способами: на основе практических соображений (разделяем целые числа на отриц-ые, полож-ые и 0) и по определенным правилам, касающимся их структуры (значение первого бита в двоичном виде целого числа). Для определения классов эквивалентных сложных данных обычно используют их структуру и классы эквивалентности данных простых типов, из которых они построены. Для документов, которые описываются некоторой контекстно свободной грамматикой, можно определить следующие метрики покрытия: \underline{\textit{метрика покрытия правил}} — доля правил грамматики, использованных для построения тестовых данных, среди всех ее правил; \underline{\textit{метрика покрытия альтернатив}} — для каждого правила определяется, сколько имеется возможных альтернатив его раскрытия, и вместо 1 в определении предыдущей метрики для всех правил учитывается это число, а для покрытых — только количество реализованных альтернатив.

Также к доменным критериям относят \underline{\textit{анализ граничных значений}} -- тестирование пределов классов эквивалентности, так как ошибки часто случаются именно на границах этих классов, \underline{\textit{попарное тестирование}} -- создание тестов для всех возможных комбинаций пар входных параметров.

\underline{\textit{3. Функциональные критерии}} -- можно определить метрику покрытия требований как долю проверяемых тестовым набором наиболее детальных выделенных пунктов требований среди тех, которые вообще можно проверить. Так, для определения \underline{\textit{метрики покрытия утверждений}} из требований выделяются элементарные проверяемые утверждения, выполнение каждого которых, вообще говоря, не связано с выполнением остальных. Метрика определяется как доля проверенных в тестах таких утверждений по отношению ко всем. Часто встречающийся случай -- оформление требований в виде \underline{\textit{набора правил}}. В этом случае метрикой покрытия правил считают долю проверенных тестами правил среди всех имеющихся. Структурные и функциональные критерии удачно дополняют друг друга -- структурные позволяют отслеживать полноту тестов по отношению к коду, а функциональные -- по отношению к требованиям

\underline{\textit{4. Проблемные критерии}} -- использующие явное указание ошибок, на обнаружение которых нацелен набор тестов, в качестве критерия его полноты. Наиболее удобный на практике способ измерения полноты тестирования на основе явных гипотез о возможных ошибках — это метод определения полноты тестов на основе \underline{\textit{обнаруженных мутантов}}. В рамках этого метода для языка программирования, на котором написана тестируемая программа, определяется достаточно полный набор операторов мутации. Каждый такой оператор изменяет текст программ, например, удаляя определенную инструкцию, вставляя новую инструкцию, заменяя переменные в выражениях на другие переменные того же типа или на константные выражения того же типа, заменяя операторы арифметических действий +, –, *, / друг на друга, заменяя операторы логических операций друг на друга и пр. Важно, что после применения любого из операторов мутации синтаксически и семантически корректная программа остается корректной. Те мутанты, которые эквивалентны по поведению исходной программе, т.е. ведут себя точно так же во всех ситуациях, выбрасывают из полученного множества мутантов. После этого используется метрика полноты тестов, определяемая как доля обнаруживаемых тестами мутантов среди оставшихся.

\textbf{Графы} в основном используются для моделирования потока управления (пример в структурных критериях) и потока данных (пример в доменных критериях). \textbf{Грамматики} могут быть применены для тестирования программ, которые обрабатывают текстовые данные в соответствии с определёнными форматами, например при тестировании ПО обрабатывающего документы, порожденные какой-либо грамматикой (пример в доменных критериях). \textbf{Логические выражения} используются для формального описания условий и ограничений системы -- можно анализировать предикаты в программе и разрабатывать тесты для проверки всех возможных исходов каждого предиката (истина/ложь) (пример в структурных критериях).

К основным \textbf{типовым критериям покрытия кода} можно отнести:

\begin{itemize}
    \item покрытие инструкций;
    \item покрытие ветвей;
    \item покрытие условий;
    \item покрытие путей.
\end{itemize}

% Взято из: https://mbt-course.narod.ru
% лекция 3
\subsection{OSN 12 Методы контроля качества ПО. Верификация и валидация. Виды
верификации.Экспертиза. Статический и динамический анализ. Формальные
методыверификации.Проверкамоделей.}

\section*{OSN 13 Интегрированные подходы построения тестов. Элементы технологии UniTESK. Программные контракты. Уточнение и формализация требований. Построение сценария теста на основе требований и заданного критерия полноты тестирования. Архитектура тестового набора UniTESK. Организация тестирования распределенных систем. Семантика чередования. Событийные контракты.}


\nsection{OSN 14 Спецификация и верификация параллельных программ. Синхронная и асинхронная параллельность. Справедливость планировщика.Темпоральная логика линейного времени (LTL). Проблема взаимного исключения процессов.}

1) смотри сюда: \url{http://sp.cmc.msu.ru/courses/vmp/kamkin_mc2018.pdf}, 
Страничка 135

Рассмотрим следующую упрощенную модель.
\textbf{Параллельной программой} \textit{P}, или просто программой, называется конечное множество последовательных программ \textit{$P_i$} над общим множеством переменных. 
Отдельная программа этого множества называется \textit{процессом}.
Переменные, используемые только в одном процессе, называются \textit{локальными}; переменные, используемые в нескольких процессах, — \textit{разделяемыми} или \textit{глобальными}.
\textit{Семантика}, или \textit{модель вычислений}, параллельной программы может быть определена, используя парадигму \textit{чередования} (интерливинга), известную также как \textbf{асинхронный параллелизм}. 
\textit{Стратегию} выбора процесса, также как и сущность, реализующую эту стратегию, мы будем называть \textbf{планировщиком}. 

При анализе свойств реагирующих систем предполагают, что \textit{планировщик} является \textbf{справедливым}, т.е. каждый процесс периодически, время от времени, выбирается для исполнения; другими словами, невозможна ситуация, когда какой-нибудь процесс не выбирается бесконечно долго.

Такая модель вычислений, известная как \textbf{синхронный параллелизм}, широко используется при проектировании цифровой аппаратуры. 
В этой модели параллельные присваивания в одну переменную либо запрещаются, либо совершается только одно из них, выбранное \textit{недетерминированным образом}.

Далее мы будем рассматривать исключительно \textit{асинхронные} параллельные программы. 
Более того, мы будем рассматривать программы, работающие в «бесконечном цикле». 
Речь идет о так называемых \textit{реагирующих}, или \textit{реактивных}, системах (от англ. reactive). 
Такие системы реагируют на события окружения, выполняя в ответ те или иные действия. 
Это обширный класс программ, включающий операционные системы, драйверы устройств, телекоммуникационные среды, системы управления и т.п.
На данном этапе под \textit{событием} понимается условие на значения разделяемых переменных (\textit{охранное условие}, или \textit{защита}), а под \textit{действием} — часть программы, срабатывающая, когда условие становится истинным. 


\paragraph{\textbf{Темпоральная логика} линейного времени (LTL, Linear-time Temporal Logic).} 
\text{}
\newline
В LTL к синтаксису классической логики высказываний добавлены два темпоральных оператора: унарный оператор \textbf{X} (от англ. ne\textbf{X}t time — в следующий момент времени) и бинарный оператор \textbf{U} (от англ. \textbf{U}ntil — до тех пор, пока не). 
Эти два оператора образуют \textbf{темпоральный базис LTL}. 
Формула логики LTL задается следующей грамматикой:
\begin{equation}
  \phi ::= p | \neg \phi | \phi \vee \phi  |  \textbf{X}\phi | \phi \textbf{U} \phi,
\end{equation}

где p - произвольное элементарное высказывание из множества элементарных высказываний.
Для удобства в формулах LTL можно использовать производные логические связки, например, $\vee$ и $\wedge$, логические константы \textit{$true$}  и  \textit{$false$}. и производные темпоральные операторы, включая \textbf{F} (от англ. in the \textbf{F}uture — когда-нибудь в будущем) и \textbf{G} (от англ. \textbf{G}lobally — глобально, всегда). На содержательном уровне темпоральные операторы интерпретируются так: 

\begin{itemize}
	\item $\textbf{X}\phi$ -- формула $\phi$ истинна в следующий момент времени.
	\item $\phi\textbf{U}\psi$ -- формула $\psi$ истинна сейчас или $\textbf{обязательно}$ будет истинна в $\textbf{будущем}$, но до этого момента (не включительно) должна быть истинна формула $\phi$.
	\item $\textbf{F}\phi \equiv \textit{true} \textbf{U} \phi$ -- формула $phi$ истинна сейчас или станет истинной когда-нибудь в будущем. 
	\item $\textbf{G}\phi \equiv \neg \textbf{F} \neg \phi$ -- формула $\neg \phi$ ложна сейчас и никогда не станет истинной в $\textit{будущем}$ ($\textit{всегда}$, начиная с настоящего момента истинна формула $\psi$).
\end{itemize}

Для более TL;DR понимания что это всё такое, см рисунок \ref{LTL-futures} ниже:

\begin{figure}[H]
  \includegraphics[width=\linewidth]{pics/ltl-timescape.png}
  \caption{LTL-futures}
  \label{LTL-futures}
\end{figure}


Параллельные программы, как и последовательные, можно верифицировать дедуктивно. 
Это справедливо как для спецификаций, заданных в форме пред- и постусловий, так и для спецификаций, выраженных на языке LTL.

Для примера обратимся к \textbf{алгоритму Петерсона} (Gary Peterson) взаимного исключения процессов. 
Пусть программа P состоит из двух процессов $P_0$ и $P_1$, работающих в бесконечном цикле. 
Периодически каждый из них входит в критическую секцию (для процесса $P_i$ критическая секция помечена меткой $CRS_i$). 
Вход процесса в критическую секцию осуществляется в два этапа: сначала он сообщает о своем намерении (метка $ SET_i$), затем ожидает возможности войти (метка $TST_i$). 
Выход процесса из критической секции осуществляется путем сброса флага $flag_i$, установленного при входе (метка $RST_i$). Описание процесса $P_i$ приведено ниже (\ref{LTL-code}):

\begin{figure}[H]
  \includegraphics[width=\linewidth]{pics/ltl-sample-code.png}
  \caption{}
  \label{LTL-code}
\end{figure}

Начальное состояние задается равенствами turn = 0 и $flag_i$, где i = 0,1. Докажем корректность приведенной реализации алгоритма Петерсона относительно следующей LTL-спецификации:

\begin{itemize}
	\item $\textbf{G}\{\neg (@CRS_0 \wedge @CRS_1)\}$ -- свойство $\textit{безопасности}$ (safety): два процесса не могут одновременно пребывать в одной критической секции.
	\item $\textbf{G}\{@SET_i \rightarrow \textbf{F} @ CRS_i\}$, где $i \in \{0, 1\}$, -- свойство $\textit{живости}$ (liveness): если процесс запросил доступ к критической секции, то рано или поздно он его получит.
\end{itemize}

При этом \textit{справедливость} планировщика, о которой говорилось выше, при помощи LTL-формул задаётся вот так:

\begin{figure}[H]
  \includegraphics[width=\linewidth]{pics/ltl-scheduler.png}
  \caption{Формулы, задающие справедливость планировщика}
  \label{LTL-scheduler}
\end{figure}     
\section*{OSN 15 Абстрактные модели: ошибки первого и второго родов (falsepositives, falsenegatives). Предикатная абстракция программ и уточнение абстракции по контрпримерам (CEGAR). Ее использование для верификации программ на языках программирования.}

\subsection{OSN 16 Информационная безопасность. Шифрование данных.
Криптографическаястойкость. Симметричная криптография. Блочный шифр
(DES) и его режимы.Ассиметричные схемы (RSA и Диффи-Хеллмана). Код
аутентификации (MAC).Цифроваяподпись(DSA).}


\nsection{OSN 17 Понятие анонимности пользователя в сети. Идентификаторы пользователя в сети на разных уровнях (устройства, ОС, ПО). Подходы к деанонимизациии, способы защиты. Концепция анонимных сетей (Mix и Tor). Луковая маршрутизация. Виды атак на анонимные сети.}

\subsubsection{Анонимность пользователя в сети}

\begin{itemize}
    \item Защита от наблюдения со стороны Интернет-провайдера (а также тех, кто его контролирует):
    – цензура
    
    – контекстная реклама
    
    – осуществление общественной, гражданской или политической деятельности
    \item Защита от наблюдения со стороны посещаемого ресурса    
\end{itemize}

\begin{itemize}
    \item Социальная анонимность – распространение (осознанное/неосознанное) в сети
персональной информации самим пользователем
    \item Техническая анонимность:
    
    – контроль за хранением и безопасностью персональных данных посредством специальных технических средств (как программных, так и аппаратных)
    
    – минимизация возможности утечки персональных данных
\end{itemize}

\textbf{Анонимность субъекта} (anonymity) -- злоумышленник не может с достаточной степенью точности идентифицировать субъект в рамках некоторого множества субъектов со схожим
набором атрибутов.

\textbf{Виды анонимности}:
\begin{itemize}
    \item Анонимность отправителя -- ни получатель сообщения, ни атакующий, наблюдающий за некоторым участком сети (между отправителем и получателем), не могут установить отправителя сообщения
    \item Анонимность получателя -- атакующий, наблюдающий за некоторым участком сети, не может установить получателя сообщения
    \item Анонимность взаимодействия -- атакующий, наблюдающий за некоторым участком сети, не может достоверно установить, от какого отправителя и какому получателю доставляется конкретное сообщение
    \item $K$-анонимность: пользователь сети не может на практике отличаться, по крайней мере, от $(K - 1)$ других пользователей сети, когда $K$ достаточно большое. Вводится для количественной оценки уровня анонимности, предоставляемого системой анонимизации отдельному пользователю
\end{itemize}

\subsubsection{Идентификаторы пользователя в сети}

Устройство пользователя:
\begin{itemize}
    \item \textit{Адресные признаки устройства}: MAC-адрес сетевого интерфейса, IP-адрес, назначенный сетевому интерфейсу
\end{itemize}

ПО пользователя:
\begin{itemize}
    \item \textit{Операционная система} (реализация стека протоколов TCP/IP): Windows 10, Ubuntu 20.04, iOS 13.2.2, ...
    \item \textit{Сетевое приложение}: User-Agent, Cookies, цифровой отпечаток, ...
\end{itemize}

\subsubsection{Подходы к деанонимизациии, способы защиты}

\textbf{Деанонимизирующие признаки ОС}: стек TCP/IP. Основаны на пробелах в
спецификациях протоколов, и, следовательно, различных реализациях в разных ОС:
\begin{itemize}
    \item IPv4: генерация значения поля ID при фрагментации, выбор начального значения TTL, установка флага DF для пакетов, не требующих фрагментации
    \item ICMP: Destination Port Unreachable, размер фрагмента не фиксирован
    \item TCP: выбор начального значения sequence number, ответы на некорректные комбинации флагов, набор опций (порядок передачи, поддержка)
\end{itemize}

Деанонимизация ОС:

\textbf{Активное сканирование:}

Злоумышленник формирует и отправляет сообщения с заданными свойствами, после чего анализирует полученные ответы. Обладает большей точностью. Может быть обнаружено системами обнаружения вторжений (IDS). Известный представитель: Nmap.

\textbf{Пассивное сканирование:}

Злоумышленник анализирует прослушиваемый трафик, ничего не отправляя в сеть. Является менее точным. Не может быть обнаружено – Известный представитель: p0f.

\textbf{Обеспечение технической анонимности}

Сокрытие факта взаимодействия между двумя узлами сети для некоторого потока сетевых пакетов:
\begin{itemize}
    \item Скрыть адресные признаки узла-отправителя и узла-получателя
    \item Скрыть идентификаторы уровня ОС
    \item Сделать цифровой отпечаток пользователя «стандартным»
\end{itemize}

Сокрытие адресных признаков: отказаться от прямой передачи данных от узла-отправителя к узлу-получателю
\begin{itemize}
    \item осуществлять передачу данных через один или несколько промежуточных узлов
\end{itemize}

\subsubsection{Концепция анонимных сетей (Mix и Tor)}

\begin{itemize}
    \item Обобщают подход на основе использования промежуточных узлов: используется цепочка из нескольких узлов
    \item Ни один промежуточный узел не знает одновременно адреса источника и адреса назначения, большинство промежуточных узлов не знает ни того, ни другого адреса
\end{itemize}

\textbf{Mix сети}:

Впервые предложены в 1981 году. Каждый узел знает только свою часть маршрута -- вложенное шифрование с использованием публичных ключей всех участвующих в передаче узлов, защита от отдельных скомпрометированных узлов. 

Mix-сети относятся к группе анонимных сетей с высокой задержкой – это ограничивает их применимость (ок для электронной почты, не ок для анонимного просмотра веб-страниц/видео).

Используется цепочка прокси-серверов, каждый из которых:

- получает сообщения от большого числа источников
    
- изменяет порядок отправки сообщений
    
- посылает в измененном порядке сообщения следующему прокси-серверу или получателю

\textbf{Tor}

Децентрализованная сеть, поддерживаемая добровольцами. Исходный код в открытом доступе. Считается одним из самых надежных способов обеспечения анонимности в сети Интернет.

Типы узлов:
\begin{itemize}
    \item \textbf{Входные узлы} -- принятие соединений, инициированных клиентами, их шифрование и перенаправление к следующему узлу
    \item \textbf{Узлы посредники} -- передача шифрованного трафика следующим узлам
    \item \textbf{Выходные узлы} -- передаточное звено между клиентом сети Tor и публичным Интернетом
    \item \textbf{Сторожевые узлы} -- клиент выбирает три узла в качестве сторожевых и использует один из них в качестве входного узла для каждой создаваемой цепочки
    \item \textbf{Мостовые узлы} входные узлы сети Tor, адреса которых не публикуются в сервере каталогов, а передаются только по запросу
    \item \textbf{Управляющие узлы} - распределены по миру, <<зашиты>> в ПО клиента, владеют информацией о состоянии всей Tor сети
\end{itemize}

\subsubsection{Луковая маршрутизация}

\begin{wrapfigure}[7]{r}{0.5\linewidth}
\vspace{-6.5ex}
\includegraphics[scale=0.18]{pics/onion-routing.png}
\end{wrapfigure}

Каждый отдельный уровень шифрования дешифруется очередным маршрутизатором. Из дешифрованных данных извлекается адрес следующего маршрутизатора. Сообщение отправляется на следующий маршрутизатор.

\subsubsection{Виды атак на анонимные сети}

\begin{enumerate}
    \item \textit{Атака на кодировку сообщения}. Если сообщение не меняет свое содержимое при прохождении по анонимной сети, весь его маршрут может быть восстановлен
    \item \textit{Атака на длину сообщения}. Если сообщение сохраняет свой размер при передаче по сети, его маршрут может быть восстановлен
    \item \textit{Атака на воспроизведение}. Атакующий может воспроизводить данные ранее переданных сообщений, ожидая, что сеть передаст эти пакеты по тому же маршруту
    \item \textit{Атака путем сговора}. Некоторые участники анонимной сети объединяются для нарушения анонимности остальных участников
    \item \textit{Атака на переполнение}. Переполнение отдельных каналы анонимной сети, чтобы сузить круг пользователей, к которым эти сообщения могут относиться
    \item \textit{Временные атаки}. Оценка продолжительности соединения путем наблюдения фактов установки и окончания соединения в различных точках входа и выхода сети
    \item \textit{Атака на объем данных}. Сопоставление общего объема передаваемых данных в точках входа и выхода сети
    \item \textit{Атака профилирования}. Долговременный анализ соединений некоторого набора пользователей -- обычно комбинация временных атак и атак на объем данных
\end{enumerate}



% --- BEGIN OF ADDITIONAL PAGE ----
\newpage  
\printbibliography
% --- END OF ADDITIONAL PAGE ---


\end{document}
