\nsection{OSN 10 Символьное выполнение: основные понятия. Схема работы системы символьного выполнения. Предикат пути, предикат безопасности. Проблема
экспоненциального взрыва, стратегии выбора следующего состояния.}

\textbf{Символьное выполнение}
\begin{itemize}
    \item <<Выполнение>> программы не на конкретных значениях входных данных, а на символьных значениях
    \item <<Выполнение>> множества путей программы одновременно: в точке ветвления, зависящей от символьных значений происходит разделение выполнения на две ветви с добавлением ограничений из условия ветвления
    \item Технология получила быстрое развитие в последнее время благодаря росту вычислительных возможностей и появлению удобных инструментов – решатель STP, общий формат уравнений \textit{SMT-LIB2}
\end{itemize}

\textbf{Представление программы}
\begin{itemize}
    \item Программа представляется в виде бинарного дерева потенциально бесконечной глубины (циклы) – дерево символьного выполнения
    \item Вершины дерева соответствуют выполнению условных переходов
    \item Рёбра – выполнению последовательных инструкций
    \item Каждый путь в дереве описывает эквивалентный класс входных данных
    \item Формула, описывающая путь – предикат пути
\end{itemize}

\textbf{Предикат пути} - набор логических формул описывающие прохождение по данному пути выполнения

\textbf{Предикат безопасности} – набор логических формул описывающие нарушение
безопасности кода (переполнение буфера)

\textbf{Схема работы}
\begin{itemize}
    \item Обход путей программы, генерация предикатов путей
    \item При обнаружении опасной ситуации (например, деление на 0) – добавление предиката безопасности
    \item Выбор очередной формулы и отправка её решателю
    \item Если формула содержала предикат безопасности и была решена – получение набора входных данных, активирующих ошибку
\end{itemize}

\textbf{Основные проблемы подхода}
\begin{itemize}
    \item Экспоненциальный взрыв – экспоненциальный рост количества путей
    \item Моделирование окружения – обработка системных/библиотечных вызовов
    \item Ограничения решателя – сложность решения уравнений
\end{itemize}

\textbf{Стратегии выбора пути}
\begin{itemize}
    \item На основе только структуры кода:
    \begin{itemize}
        \item Обход в глубину – может <<зациклится>> в цикле
        \item Обход в ширину – очень медленно доходит до содержательного кода при наличии большого количества ветвей
    \end{itemize}
    \item  На основе покрытия кода – выбирать пути с непосещенными инструкциями, или те которые посещались меньше. Позволяет обнаруживать ошибки на редко выполняемых путях, однако может не достигать некоторых инструкций никогда.
    \item Случайный выбор. Возможности:
    \begin{itemize}
        \item Всегда выбирать путь случайно
        \item Выбирать случайно если долгое время ничего не находится (гибрид)
        \item  Выбирать случайно в случае равного приоритета путей (гибрид)
    \end{itemize}
\end{itemize}