osn 1. Основные понятия дедуктивной верификации. Методы доказательства корректности программ.

osn 2.  Основные понятия дедуктивной верификации. Методы доказательства завершимости программ.

osn 3. Основные сведения об объектном языке ограничений (OCL): состав OCL-выражения,навигация по ассоциациям, виды коллекций, операции с коллекциями, учёт наследования в выражениях и наследование ограничений. Примеры использования OCL.

osn 4. Способы объектно-реляционного отображения для классов и атрибутов, бинарных и N-арных ассоциаций, классов ассоциаций, иерархий наследования. Примеры применения этих способов. Моделирование схемы реляционной базы данных с помощью диаграммы классов.

osn 5. Образцы (паттерны) проектирования, их классификация и способ описания. Примеры образцов: структурного, поведенческого и порождающего.

osn 6. Основные понятия безопасности информации: конфиденциальность, целостность,доступность. Виды защиты информации. Модель Белла-Лападулы. Понятие ошибки, уязвимости в программном обеспечении, примеры.

osn 7. Ошибка типа «переполнение буфера». Выполнение произвольного кода на исполнимом стеке. Противодействие выполнению кода на стеке: «канарейка», DEP. Выполнение произвольного кода на неисполнимом стеке. Return-to-libc, return-orientedprogramming(ROP).

osn 8. Статический анализ исходного кода с целью поиска ошибок. Типы обнаруживаемых ошибок. Путь распространения ошибки: source, propagation, sink. Потоковая и контекстная чувствительность. Качество результата анализа: false/truepositive/negative. Интерпретация результатов анализа.

osn 9. Применение отладки для оценки возможности эксплуатации уязвимостей. Технологии отладки. Отладка пользовательского кода. Полносистемная отладка ввиртуальной машине. Статическое и динамическое инструментирование. Фаззинг. Разновидности фаззинга: черный ящик, белый ящик, серый ящик.

osn 10. Символьное выполнение: основные понятия. Схема работы системы символьного выполнения. Предикат пути, предикат безопасности. Проблема экспоненциального взрыва, стратегии выбора следующего состояния.

osn 11. Критерии полноты тестирования. Доменные, функциональные, структурные и проблемные критерии полноты. Использование графов, грамматик и логических выражений для построения критериев полноты тестирования. Типовые критерии покрытия кода.

osn 12. Методы контроля качества ПО. Верификация и валидация. Виды верификации. Экспертиза. Статический и динамический анализ. Формальные методы верификации. Проверка моделей.

osn 13. Интегрированные подходы построения тестов. Элементы технологии UniTESK. Программные контракты. Уточнение и формализация требований. Построение сценария теста на основе требований и заданного критерия полноты тестирования. Архитектура тестового набора UniTESK. Организация тестирования распределенных систем. Семантика чередования.
Событийные контракты.

osn 14. Спецификация и верификация параллельных программ. Синхронная и асинхронная параллельность. Справедливость планировщика. Темпоральная логика линейного времени(LTL). Проблема взаимного исключения процессов.

osn 15. Абстрактные модели: ошибки первого и второго родов (false positives,false negatives). Предикатная абстракция программ и уточнение абстракции по контрпримерам (CEGAR). Ее использование для верификации программ на языках программирования.

osn 16. Информационная безопасность. Шифрование данных. Криптографическая стойкость. Симметричная криптография. Блочный шифр (DES) и его режимы. Ассиметричные схемы (RSA и Диффи-Хеллмана). Код аутентификации (MAC). Цифровая подпись(DSA).

osn 17. Понятие анонимности пользователя в сети. Идентификаторы пользователя в сети на разных уровнях (устройства, ОС, ПО). Подходы к деанонимизациии, способы защиты. Концепция анонимных сетей (Mix и Tor). Луковая маршрутизация.Виды атак на анонимные сети.