\nsection{OSN 14 . Проблемы статического анализа объектно-ориентированных языков (C++, Java). Поток управления в присутствии исключений. Вызовы по указателю и их анализ. Понятие одевиртуализации.}

Проблемы:
\begin{itemize}
    \item Использование динамического полиморфизма означает, что определенный метод может вызываться для разных объектов, и это может быть известно только во время выполнения. Это затрудняет точный анализ кода на этапе компиляции.
    
    \item Также в С++ есть проблемы, связанные с использованием указателей и ссылок, которые могут привести к неопределенному поведению во время выполнения. Такие проблемы могут быть трудными для обнаружения статическим анализом.
    \item В Java, статический анализ также имеет свои ограничения относительно анализа потока управления. В частности, использование динамической загрузки классов может привести к тому, что методы и поля будут известны только во время выполнения
\end{itemize}

В присутствии исключений поток управления может значительно изменится. Требуется рассматривать несколько вариантов развития событий одновременно и анализировать несколько путей.

Проблемы с вызовом по указателю в том, что достоверно может быть не известен адрес, по которому мы будем переходить. Мы можем только попытаться как-то сузить в общем случае область переходов с помощью анализа указателей. Естественно восстановить гарантировано поток управления в этом случае мы не можем.

\textbf{Одевиртуализация} (\textbf{деобфускация}) в статическом анализе - это процесс обратного преобразования двоичного кода в исходный код, который был замаскирован, возможно, с помощью техник обфускации.

Обфускация используется для защиты программного кода от анализа и понимания его работы. Однако, в некоторых случаях, необходимо провести анализ защищенного кода, например, для обнаружения уязвимостей или создания более эффективных алгоритмов.

В статическом анализе одевиртуализация может быть выполнена с помощью специальных инструментов, которые выявляют и извлекают скрытый код из программы. Эти инструменты работают на уровне машинного кода, выполняя различные декодирования и дешифрования, чтобы получить исходный код программы. Однако, этот процесс может быть сложным и затратным, особенно если оригинальный код был сильно обфусцирован.