\nsection{OSN 1-2 Основные понятия дедуктивной верификации. Методы доказательства завершимости программ.}
\paragraph{Математическая модель требований}
Математическую модель требований к верифицируемой про- грамме мы будем называть \textbf{спецификацией программы}. Спецификацией $\Phi$ программы над переменными V мы будем называть два предиката:
\begin{itemize}
    \item входной предикат $\varphi: D_x \rightarrow \{ T, F \}$
    \item выходной предикат $\psi: D_x \times D_z \rightarrow \{T, F\}$
\end{itemize}
Выходной предикат (или постусловие) определяет, какие значения выходных переменных программы являются допустимыми (правильными) относительно значений входных переменных. А входной предикат (или предусловие) определяет, при каких значеиях входных переменных требуется выполнение ограничений, описанных в выходном предикате.
\paragraph{Корректность программ}
Пусть программа задана своей моделью в виде блок-схемы P, а ее спецификация $\Phi$ – предикатами $\varphi$ и $\psi$. Мы будем говорить, что
\begin{enumerate}
    \item Программа P \textbf{частично} корректна относительно $\varphi$ и $\psi$, если для любого вектора значений входных переменных $x$, такого что  $\varphi(x) = T$ и $M[P](x) \neq \omega$, выполнено ограничение $\psi(x, M[P] (x)) = T$. Частичную корректность программы P относительно $\varphi$ и $\psi$ мы будем обозначать \{$\varphi$\}P\{$\psi$\}.
    \item Программа P \textbf{полностью} корректна относительно $\varphi$ и $\psi$, если для любого вектора значений входных переменных $x$, такого что $\varphi(x) = T$, выполнены ограничения $M[P](x) \neq \omega$ и $\psi(x, M[P] (x)) = T$. Полную корректность программы P относительно $\varphi$ и $\psi$ мы будем обозначать $\langle\varphi\rangle$P$\langle\psi\rangle$.

\end{enumerate}

Доказательство полной корректности проводится в два этапа. Сначала доказывается частичная корректность программы. Для этого используется \textbf{метод индуктивных утверждений Флойда}. Вторым шагом проводится доказательство завершаемости программы. Для решения используется \textbf{метод фундированных множеств Флойда}.

\paragraph{Корректность программ. Метод индуктивных утверждений Флойда}

Если $\alpha$ - путь, то $R_{\alpha}(x, y) : D_x \times D_y \rightarrow \{ T, F \}$ - \textit{предикат допустимости пути}. $r_{\alpha}(x,y): D_x \times D_y \rightarrow D_y$ - \textit{функция пути}


Предикат (допустимости) пути $\alpha R_{\alpha}(x, y) \rightarrow \{T,F\}$ определяет, какими должны быть переменные в начале пути, чтобы дальнейшее выполнение шло по этому пути.
Функция пути $r_\alpha(x,y)$ определяет, как изменяются промежуточные переменные в результате выполнения пути.
Для определения этих штук используется метод обратных подстановок:


\begin{enumerate}
    \item Выбираются дуги графа - точки сечения. Каждый цикл должен содержать хотя бы одну точку сечения.
    \item Начальный базовый путь - от START к первой точке сечения; Пути между точками сечения - базовые пути; Пути, завершающиеся в HALT - конечный базовый путь; Путь от START до HALT без т.с. - простой базовый путь.
    \item Для каждой точки сечения определяем путь $\pi(x, y)$, который характеризует отношение между переменными при прохождении данной связки - индуктивные утверждения. Входной предикат - к START, Выходной - ко всем HALT.
    \item Для каждого базового пути от $i$ к $j$ формулируем условия верификации: 
    
    $\forall x \in D_x \forall y \in D_y $  $[\varphi(x) \And p_i(x, y) \And R_{\alpha}(x, y) \Rightarrow p_j(x, r_{\alpha}(x, y)) ]$ (т. е. если $p_i(x, y)$ истинен и $x, y$ такие, что мы идем по пути, то $p_j(x, r_{\alpha}(x, y))$ истинен). 
    
    Для начального базового пути: $\forall x \in D_x [\varphi(x) \And R_{\alpha}'(x) \Rightarrow p_j(x, r_{\alpha}'(x))]$
    
    Для конечного базового пути: $\forall x  \in D_x \forall y \in D_y [\varphi(x) \And p_i(x, y) \And R_{\alpha}''(x, y) \Rightarrow \psi(x, r_{\alpha}''(x, y))]$
    
    \item Если все условия верификации истинны, то блок-схема частично корректна относительно спецификации. Т.е. нужно доказать истинность всех условий.

\end{enumerate}
 

\paragraph{Завершаемость программ. Метод фундированных множеств Флойда}
Основан на аппарате частично-упорядоченных множеств (множеств с операцией сравнения - транзитивной, асимметричной, иррефлексивной)
Частично-упорядоченное множество - фундированное, если нет бесконечно убывающей последовательности эл-тов (мн-во натуральных чисел)
Метод:
\begin{enumerate}
    \item Выбираются точки сечения, чтобы каждый цикл разбивался, фундированное множество W, индуктивные утверждения для каждой точки сечения.
    \item Для каждой точки сечения определяются оценочные функции $u_i(x, y) D_x * Dy \rightarrow W_i$. Оценочная функция должна отвечать условию корректности: 
    
    $\forall x  \in D_x \forall y \in D_y [\varphi(x) \And q_i(x, y) \Rightarrow u_i(x, y) \in W]$
    \item Для каждого промежуточного базового пути формулируется условие завершимости (если индуктивное утверждение истинно и выполнение идет по пути, то оценочная функция в конце пути меньше оценочной функции в конце пути):
    
    $\forall x  \in D_x \forall y \in D_y [\varphi(x) \And q_i(x, y) \And R_{\alpha}(x, y) \Rightarrow (u_i(x, y) \succ u_j(x, r_{\alpha}(x, y)))]$
    
    \item Условия успешности вычислений. Для всех операторов результат вычислений/присваиваний не приводит к ошибке.

\end{enumerate}


