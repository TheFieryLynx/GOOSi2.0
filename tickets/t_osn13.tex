\nsection{OSN 13 Интегрированные подходы построения тестов. Элементы технологии UniTESK..}


\textbf{Интеграционное тестирование} проверяет взаимодействие модулей между собой, включая обмен данными и вызовы операций, с учетом ограничений, таких как предусловия вызовов. Оно может применяться на различных этапах разработки и использует разные стратегии интеграции модулей. Рекомендуется использовать смешанные стратегии, избегая проведения тестирования сразу для всех модулей большой системы без предварительной отладки взаимодействия внутри отдельных групп модулей.

При стратегии \textbf{«сверху вниз»} тестируются модули с самого верхнего уровня к нижнему, добавляя модули по мере необходимости, используя заглушки на промежуточных этапах.

При стратегии \textbf{«снизу вверх»} тестируются модули с нижнего уровня к верхнему, минимизируя использование заглушек.

\textit{Тестовые заглушки} используются для замены неразработанных компонентов в процессе интеграционного и модульного тестирования. Эти заглушки имеют простой интерфейс, который позволяет заменить отсутствующие компоненты, обеспечивая работоспособность тестируемого компонента. Основное отличие от модульного тестирования заключается в том, что интеграционное тестирование проверяет именно взаимодействие модулей, а не сами модули.


Разработка тестов и тестирование на основе моделей требует явного формулирования используемых моделей. \textbf{Технология UniTESK}, разработанная в Институте системного программирования РАН в 1995-2002 годах, предназначена для создания и поддержки наборов тестов для сложных систем с обширными функциями, сложным интерфейсом и большим объемом кода. Она предоставляет методы разработки тестов, архитектуру тестового набора, набор техник построения и организации тестов, языки моделирования и инструменты для работы с моделями и создания тестов из них. Важно отметить, что эта технология эффективна в поддержке и расширении тестов на протяжении длительного периода, но может быть менее эффективной без выполнения определенных условий, таких как появление новых версий системы и длительное обслуживание тестов.


\textbf{Модульное тестирование} проверяет отдельные модули независимо от окружения, основываясь на программных контрактах, которые описывают условия для входных и выходных данных, а также целостность внутренних данных модуля. Это важная часть отладочного тестирования, которое проводят разработчики для проверки своего кода.

\textbf{Программные контракты} представляют собой описание компонентов с определенным интерфейсом, включая структуру данных, типы и инварианты. Каждая операция имеет предусловие и постусловие, которые определяют условия вызова операции и требования к ее результатам. Это сочетание описывает состояния компонента и его операции, а также их поведение, что облегчает тестирование в среде, где нельзя выполнить определенные операции.

При определении требований к системе необходимо выявить их источники, собрать и систематизировать их в виде модели поведения. Этот процесс включает следующие этапы:

\begin{itemize}
    \item Определение источников требований, включая документацию, разработчиков, аналитиков, стандарты и другие системы.
    \item Выделение и сбор требований, каждое из которых должно быть связано с источником.
    \item Систематизация требований, назначение уникальных идентификаторов, выявление связей и классификация по обязательности выполнения.
    \item Уточнение, согласование и устранение противоречий и неполноты, при необходимости привлекая заинтересованных лиц.
    \item Финальная декомпозиция интерфейса системы и внесение изменений в соответствии с полученной информацией.
    \item Формализация требований в виде модели поведения, используя расширенные автоматы, программные контракты и другие подходы.

\end{itemize}

Эти этапы помогают разработчикам точно определить требования к системе и обеспечить их соответствие.

При разработке и выполнении тестов используется следующий метод:
\begin{itemize}
    \item Определение целей и рамок проекта.
    \item Анализ требований к тестируемой системе и к полноте тестирования.
    \item Разработка тестов и их выполнение.
    \item Анализ результатов тестирования.
\end{itemize}

Эти шаги могут выполняться последовательно или итеративно. Разработка и выполнение тестов включает в себя несколько этапов:

\begin{itemize}
    \item \textit{Выбор общей схемы теста и определение видов тестов}. Определяется стратегия построения тестов в соответствии с требованиями к системе. Включает выбор типов тестов и их целенаправленное применение.
    \item \textit{Определение дополнительных проверок, если требуется}. Выявляются дополнительные аспекты тестирования, выходящие за рамки базовых проверок, и определяются соответствующие методы проверки.
    \item \textit{Разработка генераторов тестовых данных, при необходимости}. Создаются инструменты для автоматизированной генерации тестовых данных, учитывающие различные сценарии использования системы.
    \item \textit{Определение состояний и действий, построение автомата теста, если нужно}. Строится автоматическая модель для описания поведения системы в различных состояниях и определения последовательности действий для тестирования.
    \item \textit{Разработка адаптеров для преобразования интерфейсов, если интерфейсы тестируемой системы отличаются от модели}. Разрабатываются промежуточные компоненты для обеспечения совместимости тестов с реальной системой и корректного выполнения тестовых сценариев.
    \item \textit{Прогоны и отладка тестов}. Осуществляется запуск и отладка тестов для проверки их корректности и эффективности в проверке системы.
\end{itemize}

При использовании автоматного теста для достижения критерия полноты тестирования используется редукция модели по критерию полноты.


\textbf{Архитектура тестового набора UniTESK} базируется на следующих принципах:

\begin{itemize}
    \item \textit{Разработка тестов на основе модели поведения системы}. Тесты создаются согласно модели поведения системы, позволяя разрабатывать их независимо от компонентов системы. Модель поведения служит источником для автоматического построения тестовых сценариев.
    \item \textit{Модель основана на требованиях и желаемых свойствах}. Модель поведения строится на основе требований и желательных характеристик системы, а не на основе проектных решений. Это позволяет легко адаптировать тестовый набор под изменения в требованиях, а не в коде системы.
    \item \textit{Использование близких к языку разработки моделей}. Модели создаются на языках программирования или их расширениях, схожих с используемыми при разработке системы, для уменьшения усилий на их понимание.
    \item \textit{Применение различных техник для построения тестов}. Тесты создаются с использованием разнообразных техник, включая нацеленное тестирование, комбинаторное построение тестовых данных и автоматные методы. Этот выбор зависит от типа тестирования, необходимой полноты, сложности данных и состояния системы, наличия параллелизма и асинхронности в ее поведении.
\end{itemize}


\textbf{Организация тестирования распределенных систем}. 

В тестировании распределенных систем, особенно сложных, где компоненты функционируют параллельно, состояние системы, влияющее на результаты работы, может зависеть не только от внутренних состояний компонентов, но и от текущего состояния среды передачи сообщений. Для создания разнообразных состояний такой системы требуются более сложные комбинации воздействий, включающие несколько последовательностей, подаваемых на параллельные входы. Важное различие между системой с одним потоком управления и системой с несколькими параллельными потоками заключается в том, что в первом случае состояние системы можно контролировать с помощью последовательных воздействий, в то время как во втором случае это состояние нельзя ни наблюдать, ни контролировать в общем случае, даже при соблюдении определенных временных интервалов.


\textbf{Семантика чередования} - техника для проверки корректности поведения компонента при параллельных воздействиях. Она позволяет определить корректность результатов работы набора параллельных вызовов, упорядочив их так, чтобы все пред- и постусловия операций выполнялись. Подразумевается, что каждая операция является атомарной и выполняется без вмешательства других. Если это не так, необходимо выделить атомарные действия и указать их в модели системы.

\textbf{Событийные контракты} в тестировании программного обеспечения помогают проверить взаимодействия между компонентами системы или системами. Они определяют ожидаемое поведение системы при различных событиях

\begin{itemize}
    \item \textit{Определение контракта}. Описывает ожидаемое поведение системы при определенных событиях, включая параметры событий и реакцию системы на них.
    \item \textit{Тестирование событий}. Симулируются события для проверки реакции системы. Это позволяет определить, соответствует ли система заданным контрактам.

    \item \textit{Валидация и верификация}. Проверяется соответствие реального поведения системы ожидаемому, выявляются возможные ошибки в обработке событий.

    \item \textit{Автоматизация тестирования}. Событийные контракты часто используются в автоматизированном тестировании для повышения эффективности и обеспечения повторяемости тестов.

    \item \textit{Документирование и коммуникация}. Служат документацией, предоставляя явное описание ожиданий и проверок, связанных с событиями, для команды разработки.
\end{itemize}