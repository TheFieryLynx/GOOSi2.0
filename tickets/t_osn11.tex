\nsection{OSN 11 Критерии полноты тестирования. Доменные, функциональные, структурные и проблемные критерии полноты. Использование графов, грамматик
и логических выражений для построения критериев полноты тестирования. Типовые критерии покрытия кода}

\textbf{Критерии полноты тестирования} 

Набор тестов крайне важно строить так, чтобы используемые тесты проверяли как можно больше разных аспектов функциональности системы в как можно большем разнообразии ситуаций -- для этого используют критерии полноты тестирования. Чаще всего для определения критерия полноты некоторые из возможных тестовых ситуаций рассматривают как эквивалентные и определяют количество классов неэквивалентных тестовых ситуаций, встретившихся или «покрытых» во время тестирования. При этом определяется и числовая метрика тестового покрытия — доля покрытых классов ситуаций среди всех возможных. Критерий полноты может использовать различные значения метрики, например, он может требовать, чтобы полный тестовый набор всегда покрывал 100\% выделенных классов ситуаций, или же считать достаточным покрытие 85\% классов ситуаций. Поскольку для одной метрики покрытия можно определить много критериев полноты, далее речь, чаще всего, идет о различных метриках тестового покрытия.

\underline{\textit{1. Структурные критерии}} -- основанные на структуре тестируемой системы. Можно выделить три уровня структурных метрик — уровень отдельной функции или отдельного метода класса, уровень компонента или класса, включающего несколько операций, и уровень подсистемы или системы в целом, в составе которых может быть много компонентов. Для одной функции: \underline{\textit{метрика покрытия инструкций}}, равная доле выполненных во время тестирования инструкций кода функции по отношению ко всем ее достижимым инструкциям; \underline{\textit{метрика покрытия ветвей}} -- это доля покрытых в ходе теста ветвей по отношению к общему количеству достижимых ветвей (получаемые из ГрафаПУ); \underline{\textit{метрика покрытия условий}} -- оценивает выполнение (истинность и ложность) всех отдельных условий в логических выражениях, каждое условие в выражении должно быть проверено на оба возможных значения (true и false) (метрика помогает обнаруживать ошибки в условиях, которые могут быть не затронуты тестами, ориентированными только на ветвление). Для полноценного тестирования ПО рекомендуется комбинировать оба подхода, чтобы обеспечивать как выполнение всех логических ветвей программы, так и проверку всех возможных значений условий.

На уровне компонентов, содержащих несколько методов, метрики покрытия могут определяться с использованием метрик покрытия кода отдельных методов. Но кроме этого есть более высокоуровневая информация — граф вызовов методов и функций компонента друг из друга. \underline{\textit{Метрика покрытия вызовов}} вычисляется как отношение выполненных различных инструкций вызова к общему количеству достижимых таких инструкций. Метрики на основе потоков данных на уровне компонента строятся на базе изменения и использования глобальных переменных или полей класса различными методами.

На уровне подсистемы графы потоков данных и потока управления во многом сливаются, превращаясь в схемы передачи управления или сообщений между компонентами. В больших системах могут использоваться метрики, основанные просто на доле затронутых тестами функций, компонентов, форм или окон, таблиц данных или других элементов данных по отношению к общему числу соответствующих элементов (просто измеряются и не требуют для понимания определения каких-то дополнительных сущностей).

\underline{\textit{2. Доменные критерии}} -- основываются на структуре входных и выходных данных. Чтобы определить метрику покрытия входных данных, нужно разбить их на подмножества эквивалентных с какой-то точки зрения данных. Проводить разделение данных можно двумя способами: на основе практических соображений (разделяем целые числа на отриц-ые, полож-ые и 0) и по определенным правилам, касающимся их структуры (значение первого бита в двоичном виде целого числа). Для определения классов эквивалентных сложных данных обычно используют их структуру и классы эквивалентности данных простых типов, из которых они построены. Для документов, которые описываются некоторой контекстно свободной грамматикой, можно определить следующие метрики покрытия: \underline{\textit{метрика покрытия правил}} — доля правил грамматики, использованных для построения тестовых данных, среди всех ее правил; \underline{\textit{метрика покрытия альтернатив}} — для каждого правила определяется, сколько имеется возможных альтернатив его раскрытия, и вместо 1 в определении предыдущей метрики для всех правил учитывается это число, а для покрытых — только количество реализованных альтернатив.

Также к доменным критериям относят \underline{\textit{анализ граничных значений}} -- тестирование пределов классов эквивалентности, так как ошибки часто случаются именно на границах этих классов, \underline{\textit{попарное тестирование}} -- создание тестов для всех возможных комбинаций пар входных параметров.

\underline{\textit{3. Функциональные критерии}} -- можно определить метрику покрытия требований как долю проверяемых тестовым набором наиболее детальных выделенных пунктов требований среди тех, которые вообще можно проверить. Так, для определения \underline{\textit{метрики покрытия утверждений}} из требований выделяются элементарные проверяемые утверждения, выполнение каждого которых, вообще говоря, не связано с выполнением остальных. Метрика определяется как доля проверенных в тестах таких утверждений по отношению ко всем. Часто встречающийся случай -- оформление требований в виде \underline{\textit{набора правил}}. В этом случае метрикой покрытия правил считают долю проверенных тестами правил среди всех имеющихся. Структурные и функциональные критерии удачно дополняют друг друга -- структурные позволяют отслеживать полноту тестов по отношению к коду, а функциональные -- по отношению к требованиям

\underline{\textit{4. Проблемные критерии}} -- использующие явное указание ошибок, на обнаружение которых нацелен набор тестов, в качестве критерия его полноты. Наиболее удобный на практике способ измерения полноты тестирования на основе явных гипотез о возможных ошибках — это метод определения полноты тестов на основе \underline{\textit{обнаруженных мутантов}}. В рамках этого метода для языка программирования, на котором написана тестируемая программа, определяется достаточно полный набор операторов мутации. Каждый такой оператор изменяет текст программ, например, удаляя определенную инструкцию, вставляя новую инструкцию, заменяя переменные в выражениях на другие переменные того же типа или на константные выражения того же типа, заменяя операторы арифметических действий +, –, *, / друг на друга, заменяя операторы логических операций друг на друга и пр. Важно, что после применения любого из операторов мутации синтаксически и семантически корректная программа остается корректной. Те мутанты, которые эквивалентны по поведению исходной программе, т.е. ведут себя точно так же во всех ситуациях, выбрасывают из полученного множества мутантов. После этого используется метрика полноты тестов, определяемая как доля обнаруживаемых тестами мутантов среди оставшихся.

\textbf{Графы} в основном используются для моделирования потока управления (пример в структурных критериях) и потока данных (пример в доменных критериях). \textbf{Грамматики} могут быть применены для тестирования программ, которые обрабатывают текстовые данные в соответствии с определёнными форматами, например при тестировании ПО обрабатывающего документы, порожденные какой-либо грамматикой (пример в доменных критериях). \textbf{Логические выражения} используются для формального описания условий и ограничений системы -- можно анализировать предикаты в программе и разрабатывать тесты для проверки всех возможных исходов каждого предиката (истина/ложь) (пример в структурных критериях).

К основным \textbf{типовым критериям покрытия кода} можно отнести:

\begin{itemize}
    \item покрытие инструкций;
    \item покрытие ветвей;
    \item покрытие условий;
    \item покрытие путей.
\end{itemize}

% Взято из: https://mbt-course.narod.ru
% лекция 3