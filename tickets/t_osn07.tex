\nsection{OSN 7 Ошибка типа «переполнение буфера». Выполнение произвольного кода на исполнимом стеке. Противодействие выполнению кода на стеке: «канарейка»,
DEP. Выполнение произвольного кода на неисполнимом стеке. Return-to-libc, return-orientedprogramming(ROP).}

\textbf{Ошибка типа «переполнение буфера»}
Программа осуществляет запись в буфер, размещенный на стеке, по неверному индексу, превышающему наибольший допустимый. Ошибка типична для языков Си и Си++, а также для ассемблера. Возможные последствия эксплуатации уязвимости: нарушение доступности — аварийное завершение или зависание программы; нарушение конфиденциальности, целостности и доступности — перехват потока управления, внедрение произвольного кода. CWE оценивает вероятность эксплуатации как очень высокую.

\textbf{Условия реализации атаки}
\begin{itemize}
    \item \textbf{Исполнимый стек:} внедряемый код размещён на стеке. Если система не позволяет выполнять код из диапазона адресов, относящегося к стеку, атака не удастся.
    \item \textbf{Относительно корректное завершение функции:} после того, как перезаписан адрес возврата и предшествующие ему значения в стеке, функция должна доработать до команды ret, чтобы выполнился внедряемый код.
    \item \textbf{Постоянные адреса: }в рамках последовательных запусков программы адреса объектов в стеке не должны меняться, т.к. иначе перезаписанный адрес возврата перестанет быть корректным, и вместо перехвата потока управления можно будет получить лишь аварийное завершение программы.
\end{itemize}

\textbf{Противодействие выполнению кода на стеке}

\textbf{Неисполнимый стек} (DEP) — технология, позволяющая помечать сегменты или страницы памяти стека как неисполняемые. При попытке передать управление на код, размещённый в такой памяти, происходит аварийное завершение процесса. Аппаратно поддерживается на уровне страничной трансляции во всех процессорах x86\_64, а также в более новых x86 и ARM. В более старых моделях x86 возможна более медленная и сопряжённая с дополнительными ограничениями на исполнимые файлы реализация с использованием сегментной трансляции.

\textbf{Канарейка }
Общая идея: проверять факт перезаписи стека непосредственно перед адресом возврата перед выходом из функции. «Канарейка» — как правило случайное значение, которое размещается на стеке перед адресом возврата. Перед выходом из функции происходит сравнение значения в стеке с исходным значением. Если значения не совпадают, программа аварийно завершается.

\textbf{Обход «канарейки» на стеке:}
\begin{enumerate}
    \item «Канарейка» проверяется только перед выходом из функции, однако перехват потока управления может быть осуществлён раньше (перезаписываемый указатель на функцию).
    \item «Канарейка» может быть перезаписана, если в функции есть ошибка CWE-123 (‘Write-what-where Condition’). Также в этом случае может быть перезаписан адрес возврата, а «канарейка» останется неизменной.
    \item Если программа содержит, например, ошибку CWE-126 (‘Buffer Over-read’), или возможны множественные попытки (brute force), то значение «канарейки» может быть извлечено из стека, после чего возможна эксплуатация переполнения буфера с известным значением «канарейки».
    \item Перехват обработчика исключения на Windows (SEH-эксплоит).
\end{enumerate}

\textbf{Обход DEP: return-to-libc}
 
Вместо передачи управления на код внутри буфера можно заменить адрес возврата на адрес известной библиотечной функции, например system из стандартной библиотеки Си. 

\textbf{Противодействие:}
\begin{enumerate}


    \item \textbf{Рандомизация адресного пространства (ASLR)} — изменение карты памяти процесса при каждом запуске (процесса или системы). Без дополнительных действий (PIE) адрес загрузки основного исполняемого файла не рандомизируется. Можно попытаться обойти защиту, используя только постоянные адреса.
    \item  \textbf{Return-oriented programming (ROP)}. \textbf{Гаджет} — последовательность команд в исполняемых секциях программы, заканчивающаяся командой RET. Возвратно-ориентированное программирование (ROP) — способ построения эксплоита, при котором полезная нагрузка формируется в виде цепочки гаджетов с известными постоянными адресами. 
\end{enumerate}