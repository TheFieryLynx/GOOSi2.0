\section*{OSN 1 Основные понятия дедуктивной верификации. Методы доказательства корректности программ.}

Целью любой верификации программы является установление соответствия программы ее требованиям. Дедуктивная верификация устанавливает это соответствие в виде логического вывода утверждения о том, что программа соответствует требованиям. При этом доказывается соответствие программы требованиям на всех входах программы.
\paragraph{Математическая модель программы}

Переменные разделяются на три типа: входные, промежуточные и выходные.

\textbf{Входные} переменные содержат исходные входные значение и никогда не меняются во время работы программы. \textbf{Промежуточные} переменные используются для хранения промежуточных результатов в процессе вычисления. \textbf{Выходные} переменные содержат значения, вычисляемые данной программой.(входные переменные будем обозначать как x1, x2, …, промежуточные как y1, y2, …, выходные как z1, z2, …).

Каждая переменная v может принимать значения из некоторого множества Dv, которое называется \textbf{доменом} переменной.

Входной домен: Dx = Dx1  Dx2  …  Dxa; Домен программы: Dy = Dy1  Dy2  …  Dyb; Выходной домен: Dz = Dz1  Dz2  …  Dzc; 

Универсальный домен D: множество значений всех возможных переменных.

Расширенный домен Dv+: домен переменной v, дополненный специальным значением \{$\omega$\}, которое не входит в универсальный домен: $D^v_+$ = Dv $\cup$ \{$\omega$\}

\paragraph{Операторы программы}
5 видов операторов программы над множеством переменных:

Начальный оператор \textbf{START}: y $\leftarrow$  f(x). Здесь f является функцией Dx $\rightarrow$ Dy+, инициализирующей промежуточные переменные программы на основе значений ее входных переменных.

Оператор присваивания \textbf{ASSIGN}: y $\leftarrow$ g(x, y). Здесь g является функцией                 Dx X Dy $\rightarrow$ $D^y_+$, вычисляющей новые значения промежуточных переменных.

Условный оператор \textbf{TEST}: t(x, y). Здесь t является предикатом на множестве значений входных и промежуточных переменных программы.

Оператор соединения  \textbf{JOIN}.

Оператор завершения  \textbf{HALT}: z $\leftarrow$ h(x, y). Здесь h является функцией Dx X  Dy $\rightarrow$ $D^z_+$, устанавливающей значения выходных переменных программы.

\includegraphics[scale=0.2]{pics/halt.png}

\textbf{Модель программы} - блок-схема. \textbf{Блок-схема} это тройка ( V, N, E ), где \textbf{V} – конечное множество переменных программы, \textbf{N} – конечное множество операторов блок-схемы, E $\subseteq$ N x \{ T, F, $\epsilon$ \} x N – конечное множество связок блок-схемы, помеченных символами T, F или $\epsilon$.

\textbf{Корректно-определенная блок-схема}:
\begin{enumerate}
    \item Ровно один начальный оператор и не менее одного завершающего оператора.
    \item Любой оператор находится на ориентированном пути от начального оператора к некоторому завершающему оператору.
    \item Число связок, выходящих из каждого оператора, и пометки этих связок соответствуют типу оператора: 
    \begin{itemize}
        \item Из начального оператора выходит ровно 1 дуга, помеченная символом $\epsilon$
        \item Из оператора присваивания выходит ровно 1 дуга, помеченная символом $\epsilon$.
        \item Из условного оператора выходит ровно 2 дуги, причем одна из них помечена символом T, а другая – символом F.
        \item Из оператора соединения выходит ровно 1 дуга, помеченная символом $\epsilon$.
        \item Из завершающего оператора не выходит ни одной дуги.
    \end{itemize}
    \item Число связок, входящих в каждый оператор, соответствует его типу
    \begin{itemize}
        \item В начальный оператор не входит ни одна дуга
        \item В оператор присваивания, условный и завершающий оператор входит ровно одна дуга
        \item В оператор соединения входит не менее одной дуги
    \end{itemize}
\end{enumerate}





