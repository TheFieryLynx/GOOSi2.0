\nsection{OSN 12 Исключение частично-избыточных выражений методом анализа потока данных.}

Избыточность в программе существует в различных формах. Она может быть в виде общих подвыражений, когда несколько вычислений выражения дают одно и то же значение. Она может быть и в виде выражения, инвариантного относительно цикла, которое вычисляет одно и то же значение на каждой итерации цикла. Избыточность может быть частичной, если она обнаруживается вдоль некоторых, но не всех, путей. Общие подвыражения и выражения, инвариантные относительно цикла, можно рассматривать как частные случаи частичной избыточности; таким образом, для устранения различных видов избыточности можно разработать единый алгоритм устранения частичной избыточности.

При устранении частичной избыточности при помощи единого алгоритма находится множество типов избыточных операций. Этот алгоритм иллюстрирует, как много задач потоков данных могут использоваться при поиске оптимального размещения выражений.

\begin{enumerate}
    \item Ограничения на размещения накладываются анализом ожидаемости вы- ражений, который представляет собой обратный анализ потока данных с пересечением в качестве оператора сбора и определяет, используется ли выражение впоследствии каждой точкой программы по всем путям.
    
    \item Наиболее ранее размещение выражения определяется точкой программы, в которой выражение ожидаемо, но не доступно. Поиск доступных выраже- ний выполняется при помощи прямого анализа потока данных с оператором сбора, представляющим собой пересечение множеств. Этот анализ выясняет, является ли выражение ожидаемым перед каждой точкой программы вдоль всех путей.
    
    \item Наиболее позднее размещение выражения определяется точкой программы, в которой выражение более не может быть откладываемым. Выражения в некоторой точке программы являются откладываемыми, если вдоль всех путей, достигающих данной точки программы, не встречается использование этого выражения. Поиск откладываемых выражений выполняется при помощи прямого анализа потока данных с оператором сбора, представля- ющим собой пересечение множеств.
    
    \item Присваивание временным переменным устраняется, если только они не используются впоследствии вдоль некоторого пути. Поиск используемых выражений выполняется при помощи обратного анализа потока данных с оператором сбора, представляющим собой объединение множеств.
\end{enumerate}

Опишем данный алгоритм:
\begin{enumerate}
    \item Анализируем граф потока управления программы и строим его модель.
    \item Для каждого блока базовых операций вычисляем множество доступных выражений.
    \item Для каждой переменной в программе определяем ее определение, использование и живучесть.
    \item Используя множества доступных выражений и информацию о живучести переменных, определяем частично-избыточные выражения.
    \item Выбираем наилучшее место для вычисления выражения и создания временной переменной для хранения результата.
    \item Заменяем частично-избыточное выражение ссылкой на временную переменную.
    \item Повторяем шаги 2-6 до тех пор, пока есть частично-избыточные выражения в программе.
\end{enumerate}