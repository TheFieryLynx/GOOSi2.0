\section*{OSN 14 Спецификация и верификация параллельных программ. Синхронная и асинхронная параллельность. Справедливость планировщика.Темпоральная логика линейного времени (LTL). Проблема взаимного исключения процессов.}

\text{}
\newline
\indent

1) смотри сюда: \url{http://sp.cmc.msu.ru/courses/vmp/kamkin_mc2018.pdf}, 
Страничка 135

Рассмотрим следующую упрощенную модель.
\textbf{Параллельной программой} \textit{P}, или просто программой, называется конечное множество последовательных программ \textit{$P_i$} над общим множеством переменных. 
Отдельная программа этого множества называется \textit{процессом}.
Переменные, используемые только в одном процессе, называются \textit{локальными}; переменные, используемые в нескольких процессах, — \textit{разделяемыми} или \textit{глобальными}.
\textit{Семантика}, или \textit{модель вычислений}, параллельной программы может быть определена, используя парадигму \textit{чередования} (интерливинга), известную также как \textbf{асинхронный параллелизм}. 
\textit{Стратегию} выбора процесса, также как и сущность, реализующую эту стратегию, мы будем называть \textbf{планировщиком}. 

При анализе свойств реагирующих систем предполагают, что \textit{планировщик} является \textbf{справедливым}, т.е. каждый процесс периодически, время от времени, выбирается для исполнения; другими словами, невозможна ситуация, когда какой-нибудь процесс не выбирается бесконечно долго.

Такая модель вычислений, известная как \textbf{синхронный параллелизм}, широко используется при проектировании цифровой аппаратуры. 
В этой модели параллельные присваивания в одну переменную либо запрещаются, либо совершается только одно из них, выбранное \textit{недетерминированным образом}.

Далее мы будем рассматривать исключительно \textit{асинхронные} параллельные программы. 
Более того, мы будем рассматривать программы, работающие в «бесконечном цикле». 
Речь идет о так называемых \textit{реагирующих}, или \textit{реактивных}, системах (от англ. reactive). 
Такие системы реагируют на события окружения, выполняя в ответ те или иные действия. 
Это обширный класс программ, включающий операционные системы, драйверы устройств, телекоммуникационные среды, системы управления и т.п.
На данном этапе под \textit{событием} понимается условие на значения разделяемых переменных (\textit{охранное условие}, или \textit{защита}), а под \textit{действием} — часть программы, срабатывающая, когда условие становится истинным. 


\paragraph{\textbf{Темпоральная логика} линейного времени (LTL, Linear-time Temporal Logic).} 
\text{}
\newline
В LTL к синтаксису классической логики высказываний добавлены два темпоральных оператора: унарный оператор \textbf{X} (от англ. ne\textbf{X}t time — в следующий момент времени) и бинарный оператор \textbf{U} (от англ. \textbf{U}ntil — до тех пор, пока не). 
Эти два оператора образуют \textbf{темпоральный базис LTL}. 
Формула логики LTL задается следующей грамматикой:
\begin{equation}
  \phi ::= p | \neg \phi | \phi \vee \phi  |  \textbf{X}\phi | \phi \textbf{U} \phi,
\end{equation}

где p - произвольное элементарное высказывание из множества элементарных высказываний.
Для удобства в формулах LTL можно использовать производные логические связки, например, $\vee$ и $\wedge$, логические константы \textit{$true$}  и  \textit{$false$}. и производные темпоральные операторы, включая \textbf{F} (от англ. in the \textbf{F}uture — когда-нибудь в будущем) и \textbf{G} (от англ. \textbf{G}lobally — глобально, всегда). На содержательном уровне темпоральные операторы интерпретируются так: 

\begin{itemize}
	\item $\textbf{X}\phi$ -- формула $\phi$ истинна в следующий момент времени.
	\item $\phi\textbf{U}\psi$ -- формула $\psi$ истинна сейчас или $\textbf{обязательно}$ будет истинна в $\textbf{будущем}$, но до этого момента (не включительно) должна быть истинна формула $\phi$.
	\item $\textbf{F}\phi \equiv \textit{true} \textbf{U} \phi$ -- формула $phi$ истинна сейчас или станет истинной когда-нибудь в будущем. 
	\item $\textbf{G}\phi \equiv \neg \textbf{F} \neg \phi$ -- формула $\neg \phi$ ложна сейчас и никогда не станет истинной в $\textit{будущем}$ ($\textit{всегда}$, начиная с настоящего момента истинна формула $\psi$).
\end{itemize}



